\documentclass[12pt,a4paper]{article}
\usepackage[width=.75\textwidth]{caption}
\usepackage{graphicx}
\usepackage{authblk}
\usepackage{amsmath}
\usepackage{amsfonts}
\usepackage{braket}
\usepackage{siunitx}
%\usepackage{mathrsfs}
\usepackage[mathscr]{euscript}
\usepackage[top=2cm, bottom=2cm, left=2cm, right=2cm]{geometry}
\usepackage{fancyhdr}

\pagestyle{fancy}
\begin{document}

%title and author details
\title{Information Theory of Matter Channels}
\author[1]{Kevin Player\footnote{kjplaye@gmail.com}}

\maketitle

\abstract{We examine the information theoretic properties of matter.  Matter is regarded in it's own proper time stream, or ``matter channel'', as a signal with a clearly defined sampling rate and channel capacity.  We show how spinners in the Dirac field can be compellingly interpreted as matter channels.  Finally, we show how to think of proper time as an emergent property of matter}

\section{The Sampling Rate of Matter}
\label{rate}
We review several results which all point to a ``sampling rate'' of matter.
\subsection{De Broglie Matter Waves}
Almost exactly 100 years ago, De Broglie famously conjectured that matter should have the same kind of wave-like properties that light has.  This was verified using electrons, and then other particles.  The resulting matter waves have a (De Broglie) frequency
\[
  f_{DB} = \frac{E}{h}
\]

\subsection{Bremermann Computational Limit}
While thinking about Heisenberg's uncertainly principle for computation, Bremermann theorized a maximum ``clock-rate'' for a computer.  The argument was to find a minimally physically relevant time resolution for a given mass/energy.  This yielded a universal constant
\[
  C_{Br} = \frac{c^2}{h} \hspace{0.5 in} (\si[per-mode=symbol]{samples\per\kilogram.\sec})
\]
in samples per kilogram per second.  If you multiply this constant by a given mass, we find
\[
 f_{Br} = m C_{Br} = \frac{mc^2}{h} = f_{DB}
\]

\subsection{Zitterbewegung}
Gregory Breit and Erwin Schrödinger studied apparent oscillations of the electron as it appears in ferminoic field theory.  These oscillations were named Zitterbewegung or ``jittery motion'' in German.  The frequency 
\[
 f_{Zitter} = \frac{4 \pi mc^2}{h}
 \]
 is off by a factor of $4 \pi$ from $f_{DB}$ and $f_{Br}$, but is otherwise in agreement.  We will come back to this in more detail in Section \ref{time}

\subsection{Samples per Second}
We consider these frequencies to be in
\begin{itemize}
 \item cycles per second ($f_{DB}$),
 \item clocks per seconds ($f_{Br}$),
 \item or field theoretic interactions per second ($f_{Zitter}$) respectively.
\end{itemize}
But more generically we will just say samples per second. 

\section{Channel Capacity}
\subsection{Bekenstein and Dynamic Range}
Bekenstein found a limiting total amount of information within a given surface area.  Given a sphere radius $R$,  mass $M$, and an entropy $H$ we have
\begin{equation}
\label{bek}
  H \le \frac{2 \pi c R M}{\hbar \ln(2)} \hspace{0.5 in} (\si[per-mode=symbol]{bits})
\end{equation}
The inequality is equality for a black hole where we say that the information is saturated.  When $H$ is less, we still have a channel with entropy given by the right hand side.  It is just only fully ``used'' when $M$ is a black hole.

Let $f$ be the sampling rate of our mass $M$.  Consider a sphere with radius given by the  the resolution of the mass, the Compton wavelength\footnote{We will see in section \ref{time} how massive particles have instantaneous velocity equal to $c$, so this is appropriate.}
\[
  R=c/f  
\]
Then the saturated version of (\ref{bek}) yields
\[
 \frac{fH}{M} = \frac{4 \pi^2 c^2}{h \ln(2)} = C_{Bek} \hspace{0.5 in} (\si[per-mode=symbol]{bits\per\kilogram.\sec})
\]
in bits per kilogram per second.

A channel capacity is made up of two things, the number of samples per second and the effective number of bits per sample, the dynamic range.  Section \ref{rate} covered the first part and now we cover the second
\begin{equation}
\label{cap}
   C_{Bek} / C_{Br} = \frac{4 \pi^2}{\ln(2)} \approx 56.96 \hspace{0.5 in} (\si[per-mode=symbol]{bits\per sample})
\end{equation}
which is approximatly 7 bytes per sample. 

So the channel capacity is approximately (7 bytes per sample) * $f_{DB}$.
\subsection{Ensembles}
Consider a system with two masses, $m_1$ and $m_2$.  The ensemble system has a mass $m_{12}$ which is the sum of masses
\[
m_{12} = m_1 + m_2.
\]
The samples of the ensemble are just the disjoint union of the samples of $m_1$ and $m_2$.   This is to say that the samples of an ensemble are made up of the samples of all of the components.  In this way, the frequencies add, and we make sense of the linear relationship between mass and frequency.  In a similar way, the channel capacity of the ensemble is the sum of channel capacities of the components.

\section{Other Connections}
\label{time}
\subsection{Proper Time Stream}
An alternative derivation to (\ref{bek}) can be found by working in the mass's frame of reference. In this frame, the only dimension is time, which we usually call proper time in general relativity.  We imagine samples as events along the proper time line, Figure \ref{timeline}.  These zero dimensional samples seperate the one dimensional past and future and can be though of as analogs of holographic screens.  The amount of information associated with these samples, as particles entering holographic screens, was computed in \cite{thrust} and matches (\ref{cap}).

\begin{figure}[h]
\centering
\includegraphics[scale=0.23]{time_line.png}
\caption{A 1-dimensional (proper) time line.  The red, yellow, and green dots are 0-dimensional events, in the past, present and future respectively.}
\label{timeline}
\end{figure}

\subsection{Tick Tock Particles}
We outline a matter channel picture for an electron in the Dirac field.  There we consider the Dirac equation which involves 4 by 4 matricies and a 4-spinor.  Ryder \cite{ryder} presents the construction by considering the Lorentz algebra
\[
\mathfrak{so(3,1)} = \mathfrak{su(2)} \times \mathfrak{su(2)}
\]
which decomposes into two components for the 2 helicities, counter clockwise and clockwise.  More specifically, the first $\mathfrak{su(2)}$ is generated by the three coordinate boosts each along with a counter-clockwise twist, whereas the second $\mathfrak{su(2)}$ is generated by the three coordinate boosts each along with a clockwise twist.  These amount to an alternative six generators than the usual three Euclidean rotations and three boosts independently.

Using the Lie algebra decomposition, the Dirac 4-spinor can then be thought of as a system of two 2-spinors with opposite and equal momenta and opposite helicity.  In \cite{penrose}, Penrose carries this farther by suggesting that there could be a physical reality behind this construction.  He proposes that the 2-spinors are moving instantaneously at the speed of light and interact regularly at the De Broglie frequency.  He named these spinors zig-zag particles, and even suggested that the 4-spinor is given mass from the two 2-spinors regularly interacting with the Higgs field and each other.

\begin{figure}[h]
\centering
\includegraphics[scale=0.23]{zig_zag.png}
\hspace{0.5 in}
\includegraphics[scale=0.16]{zig_zag_higgs.png}
\caption{These figures was copied from \cite{penrose}.  They show his interpretation of the Dirac equation and the Higgs mechanism for an electron.}
\label{zigzag}
\end{figure}

If this mechanisim is the origin of mass, even for just electrons, we motivate an origin of time as well.  The argument is that time is fundamentaly made up of a series of events.  The 2-spinors are moving at the speed of light, so the don't interact except at the regular intervals.  The series of events in a series of samples.  The only capacity for interaction occurs at these samples and at a fixed rate.  We suggest that the zig-zag particles could have a more suggestively name as ``tick-tock'' particles since time a a series of events emerges from this picture.

\section{Prediction}
We predict that a massive particle should be limited by its channel capacity.  It should not be able to receive, store, or transmit information beyond this limitation.

\bibliographystyle{ieeetr}
\bibliography{bibliography}

\end{document}
